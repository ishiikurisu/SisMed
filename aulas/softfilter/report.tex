\documentclass[12pt, a4paper, twoside]{article}
\usepackage[utf8]{inputenc}
\usepackage[cm]{fullpage}
\usepackage{fancyhdr}
\usepackage{textcomp}
\usepackage{graphicx}

\begin{document}

\title{Lista de exercícios sobre filtros de software}
\author{Cristiano Silva Júnior}
\date{23 de Junho de 2017}
\maketitle

\section{Filtro de média}

\subsection{Mostrar a derivação da forma recursiva.}

A partir da definição da média
$$\bar{x_k}=\frac{1}{k}\sum^{k}_{i=1}{x_i}$$
Podemos fazer a seguinte manipulação
$$\bar{x_k}=\frac{1}{k-1}\sum^{k-1}_{i=1}{x_i}+\frac{x_k}{k}$$
$$\bar{x_k}=(\frac{1}{k-1}\sum^{k-1}_{i=1}{x_i})\frac{k-1}{k}+\frac{x_k}{k}$$
Como
$$\bar{x_{k-1}}=\frac{1}{k-1}\sum^{k-1}_{i=1}{x_i}$$
Então
$$\bar{x_k}=\bar{x_{k-1}}\frac{k-1}{k}+\frac{x_k}{k}\box$$

\subsection{Implementação em MATLAB}

Para implementar o filtro dado em MATLAB, foi escrito o seguinte script:

% TODO Add MATLAB script here

\section{Filtro de média móvel}

\subsection{Derivação da fórmula recursiva}

Seguindo da definição de média móvel com janela de tamanho $n$,
$$\bar{x_k}=\frac{x_{k-n+1}+x_{k-n+2}+\ldots+x_{k-1}+x_{k}}{n}$$
É possível afirmar que
$$\bar{x_k}=\frac{x_{k-n}}{n} + \frac{x_{k-n+1}+x_{k-n+2}+\ldots+x_{k-1}+x_{k}}{n} -\frac{x_{k-n}}{n}$$
$$\bar{x_k}=\bar{x_{k-1}} + \frac{x_k}{n} -\frac{x_{k-n}}{n}\box$$


\subsection{Implementação em MATLAB}

Para implementar o filtro dado em MATLAB, foi escrito o seguinte script:

% TODO Add MATLAB script here

\section{Filtro de média móvel ponderada}

\subsection{Derivação da fórmula recursiva}

% TODO Adicionar prova aqui


\subsection{Implementação em MATLAB}

Para implementar o filtro dado em MATLAB, foi escrito o seguinte script:

% TODO Add MATLAB script here



\end{document}
